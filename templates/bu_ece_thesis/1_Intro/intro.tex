\chapter{Introduction}
\label{chapter:Introduction}
\thispagestyle{myheadings}

\section{A few remarks before you start}
\label{sec:history}

Please read the short pointers below and on the subsequent pages; this will help
you avoid frustrations when submitting the final dissertation to the library.

Your thesis should have 1.5in left and top margins, and 1in right and bottom
margins. Getting this right is tricky since it may depend on your particular
Latex installation. Most likely you will need to adjust some of the dimensions
set up at the beginning of "bu\_ece\_thesis.sty" in this folder. Basically,
every installation should have the base margin of 1in at the left and top, but
this is not always the case. For example, the TexStudio/MiKTeX installation this
document was set up on, has the default top margin of 0.3125in and so an
additional margin of 0.6875in was added via $\backslash${topmargin}. In order to
adjust these dimensions, you may want to follow these steps:

\begin{itemize}
	\item compile the document into PDF,
	\item open the document in Acroread, set it to full-page viewing and
		magnification to 100\%
	\item navigate to a "full" page with the text extending from the very
		top to the very bottom and full-width left to right,
	\item measure the margins and adjust accordingly,
	\item if you are planning to print a hardcopy, you need to make sure
		to select "Page scaling" to "None" in Acrobat.
\end{itemize}

Another issue that BU librarians may complain and you are likely to encounter
are long URLs or other unbreakable text. In case of long URL addresses, you
should use the URL package; please see suitable documentation on-line.

However, if you encounter a long unbreakable word (e.g., foreign) the URL
package does not help. Have a look at the example extending into the page
margin:

\bigskip

{\it Consider the following Java-JDT plugin name in German: "`Plugin-Entwicklungsumgebung"'.}

\bigskip

Clearly, this is a problem, and BU librarians will complain. One way of fixing
this issue is to enclose the offending paragraph in {\tt
	$\backslash$begin\{sloppypar\}} and {\tt $\backslash$end\{sloppypar\}},
resulting in the following outcome:

\bigskip

\begin{sloppypar}
	{\it Consider the following Java-JDT plugin name in German:
		"`Plugin-Entwicklungsumgebung"'.}
\end{sloppypar}

\bigskip

Indeed, although the paragraph spacing becomes sloppy, at least you can hand in
the thesis!


LaTeX has a steep learning curve. You can use the original book by Lamport to
learn more \cite{lamport1985:latex}, but there are many on-line resources with
excellent instructions and examples. Just Google a LaTeX topic you would like to
explore.

As far as editing and compilation of LaTeX sources, if you have not found one
yet, TexStudio seems to be quite popular.