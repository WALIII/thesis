% ABSTRACT

While we can learn to produce stereotyped movements and maintain this ability for years, it is unclear how populations of individual neurons change their firing properties to coordinate these skills. This has been difficult to address because there is a lack of tools that can monitor populations of single neurons in freely behaving animals for the durations required to remark on their tuning. 

This thesis is divided into two main directions- device engineering and systems neuroscience. The first section describes the development of an electrode array comprised of tiny self-splaying carbon fibers that are small and flexible enough to avoid the immune response that typically limits electrophysiological recordings. I also describe the refinement of a head-mounted miniature microscope system, optimized for multi-month monitoring of cells expressing genetically encoded calcium indicators in freely behaving animals. In the second section, these tools are used to answer basic systems neuroscience questions in an animal with one of the most stable, complex learned behaviors in the animal kingdom: songbirds. This section explores the functional organization and long-term network stability of HVC, the songbird premotor cortical microcircuit that controls song. 

Our results reveal that neural activity in HVC is correlated with a length scale of $100\mu$m. At this mesocopic scale, basal-ganglia projecting excitatory neurons, on average, fire at a specific phase of a local 30Hz network rhythm. These results show that premotor cortical activity is inhomogeneous in time and space, and that a mesoscopic dynamical pattern underlies the generation of the neural sequences controlling song.  At this mesoscopic level, neural coding is stable for weeks and months. These ensemble patterns persist after peripheral nerve damage, revealing that sensory-motor correspondence is not required to maintain the stability of the underlying neural ensemble. However, closer examination of individual excitatory neurons reveals that the participation of cells can change over the timescale of days- with particularly large shifts occurring over instances of sleep. Our findings suggest that  \emph{fine-scale drift of projection neurons, stabilized by mesoscopic level dynamics dominated by inhibition, forms the mechanistic basis of memory maintenance and and motor stability.} 
