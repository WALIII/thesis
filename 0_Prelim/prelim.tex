% This file contains all the necessary setup and commands to create
% the preliminary pages according to the buthesis.sty option.

\title{Network Mechanisms Underlying Stable Motor Actions}

\author{William A. Liberti III}

% Type of document prepared for this degree:
%   1 = Master of Science thesis,
%   2 = Doctor of Philisophy dissertation.
%   3 = Master of Science thesis and Doctor of Philisophy dissertation.
\degree=2

\prevdegrees{B.S. Boston University, 2012}

\department{Department of Biology}

% Degree year is the year the diploma is expected, and defense year is
% the year the dissertation is written up and defended. Often, these
% will be the same, except for January graduation, when your defense
% will be in the fall of year X, and your graduation will be in
% January of year X+1
\defenseyear{2017}
\degreeyear{2017}

% For each reader, specify appropriate label {First, Second, Third},
% then name, and title. IMPORTANT: The title should be:
%   "Professor of Electrical and Computer Engineering",
% or similar, but it MUST NOT be:
%   Professor, Department of Electrical and Computer Engineering"
% or you will be asked to reprint and get new signatures.
% Warning: If you have more than five readers you are out of luck,
% because it will overflow to a new page. You may try to put part of
% the title in with the name.
\reader{First}{Timothy Gardner, Ph.D.}{Associate Professor of Biology}
\reader{Second}{Ian Davison, Ph.D.}{Assistant Professor of Biology }
\reader{Third}{Alberto Cruz-Martin, Ph.D.}{Assistant Professor of Biology }
%\reader{Fourth}{Jeffrey Gavornik, PhD}{Assistant Professor of Biology }
%\reader{Fifth}{Christopher Harvey, PhD}{Assistant Professor of Neurobiology, Harvard Medical School  }

% The Major Professor is the same as the first reader, but must be
% specified again for the abstract page. Up to 4 Major Professors
% (advisors) can be defined. 
\numadvisors=1
\majorprof{Timothy Gardner, Ph.D., Associate Professor of Biology }{}
%\majorprofb{First M. Last, PhD}{{Professor of Computer Science}}
%\majorprofc{First M. Last, PhD}{{Professor of Astronomy}}
%\majorprofd{First M. Last, PhD}{{Professor of Biomedical Engineering}}

%%%%%%%%%%%%%%%%%%%%%%%%%%%%%%%%%%%%%%%%%%%%%%%%%%%%%%%%%%%%%%%%  

%                       PRELIMINARY PAGES
% According to the BU guide the preliminary pages consist of:
% title, copyright (optional), approval,  acknowledgments (opt.),
% abstract, preface (opt.), Table of contents, List of tables (if
% any), List of illustrations (if any). The \tableofcontents,
% \listoffigures, and \listoftables commands can be used in the
% appropriate places. For other things like preface, do it manually
% with something like \newpage\section*{Preface}.

% This is an additional page to print a boxed-in title, author name and
% degree statement so that they are visible through the opening in BU
% covers used for reports. This makes a nicely bound copy. Uncomment only
% if you are printing a hardcopy for such covers. Leave commented out
% when producing PDF for library submission.
%\buecethesistitleboxpage

% Make the titlepage based on the above information.  If you need
% something special and can't use the standard form, you can specify
% the exact text of the titlepage yourself.  Put it in a titlepage
% environment and leave blank lines where you want vertical space.
% The spaces will be adjusted to fill the entire page.
\maketitle
\cleardoublepage

% The copyright page is blank except for the notice at the bottom. You
% must provide your name in capitals.
\copyrightpage
\cleardoublepage

% Now include the approval page based on the readers information
\approvalpage
\cleardoublepage

% Here goes your favorite quote. This page is optional.
\newpage
%\thispagestyle{empty}
\phantom{.}
\vspace{4in}

\begin{singlespace}
\begin{quote}
  \textit{Te occidere possunt sed te edere non possunt nefas est}\\

\end{quote}
\end{singlespace}

% \vspace{0.7in}
%
% \noindent
% [The descent to Avernus is easy; the gate of Pluto stands open night
% and day; but to retrace one's steps and return to the upper air, that
% is the toil, that the difficulty.]

\cleardoublepage

% The acknowledgment page should go here. Use something like
% \newpage\section*{Acknowledgments} followed by your text.

%% TO DO: finish acknolegements 
%%%%%%%%%%%%
\addcontentsline{toc}{chapter}{\numberline{}Acknowledgements}%
\newpage
\section*{\centerline{Acknowledgments}}








This body of work is comprised of a few bricks of success held together by a hidden mortar made from countless failures. Every result described in this thesis was accomplished with the help and support of fellow lab mates and collaborators. Especially I would like to thank Jeff Markowitz, Greg Guitchounts, and Daniel Leman.

I received tremendous support from my thesis committee- especially Ian Davison and Alberto Cruz-Martin. I want to thank my advisor Tim Gardner for teaching me how to think- really, helping me learn how to exercise some control over how and what to think, to balance exploration with being a finisher. Being conscious and aware enough to choose what to pay attention to and to choose how to construct meaning from experience. 

I want to thank my friends in Biology, and the Graduate program in Neuroscience- to Shelly and Sandi especially- for letting a wayward soul into the fold. 

Finally, I am blessed to have an amazing, talented family who not only provided unwavering support, but also surprisingly fruitful scientific collaborations and insight. 

\vskip 1in

%\noindent
%%Janusz Konrad\\
%Professor\\
%ECE Department
\cleardoublepage
%%%%%%%%%%%%

% The abstractpage environment sets up everything on the page except
% the text itself.  The title and other header material are put at the
% top of the page, and the supervisors are listed at the bottom.  A
% new page is begun both before and after.  Of course, an abstract may
% be more than one page itself.  If you need more control over the
% format of the page, you can use the abstract environment, which puts
% the word "Abstract" at the beginning and single spaces its text.
\addcontentsline{toc}{chapter}{\numberline{}Abstract}%
\begin{abstractpage}
% ABSTRACT

While we can learn to produce stereotyped movements and maintain this ability for years, it is unclear how populations of individual neurons change their firing properties to coordinate these skills. This has been difficult to address because there is a lack of tools that can monitor populations of single neurons in freely behaving animals for the durations required to remark on their tuning. 

This thesis is divided into two main directions- device engineering and systems neuroscience. The first section describes the development of an electrode array comprised of tiny self-splaying carbon fibers that are small and flexible enough to avoid the immune response that typically limits electrophysiological recordings. I also describe the refinement of a head-mounted miniature microscope system, optimized for multi-month monitoring of cells expressing genetically encoded calcium indicators in freely behaving animals. In the second section, these tools are used to answer basic systems neuroscience questions in an animal with one of the most stable, complex learned behaviors in the animal kingdom: songbirds. This section explores the functional organization and long-term network stability of HVC, the songbird premotor cortical microcircuit that controls song. 

Our results reveal that neural activity in HVC is correlated with a length scale of $100\mu$m. At this mesocopic scale, basal-ganglia projecting excitatory neurons, on average, fire at a specific phase of a local 30Hz network rhythm. These results show that premotor cortical activity is inhomogeneous in time and space, and that a mesoscopic dynamical pattern underlies the generation of the neural sequences controlling song.  At this mesoscopic level, neural coding is stable for weeks and months. These ensemble patterns persist after peripheral nerve damage, revealing that sensory-motor correspondence is not required to maintain the stability of the underlying neural ensemble. However, closer examination of individual excitatory neurons reveals that the participation of cells can change over the timescale of days- with particularly large shifts occurring over instances of sleep. Our findings suggest that  \emph{fine-scale drift of projection neurons, stabilized by mesoscopic level dynamics dominated by inhibition, forms the mechanistic basis of memory maintenance and and motor stability.} 

\end{abstractpage}
\cleardoublepage

% Now you can include a preface. Again, use something like
% \newpage\section*{Preface} followed by your text

% Table of contents comes after preface
\addcontentsline{toc}{chapter}{\numberline{}Table of Contents}%
\renewcommand{\contentsname}{ \centering Table of Contents}
\tableofcontents
\cleardoublepage




% If you do not have tables, comment out the following lines
%\newpage
%\listoftables
%\cleardoublepage

% If you have figures, uncomment the following line
\addcontentsline{toc}{chapter}{\numberline{}List of Figures}%
\renewcommand{\listfigurename}{ \centering List of Figures}
\newpage
\listoffigures
\cleardoublepage

% List of Abbrevs is NOT optional (Martha Wellman likes all abbrevs listed)
\addcontentsline{toc}{chapter}{\numberline{}List of Abbreviations}%
\chapter*{\centering List of Abbreviations}

\bgroup
\def\arraystretch{1.}%  1 is the default, change whatever you need

\begin{center}
  \begin{tabular}{lll}

    \hspace*{2em} & \hspace*{1in} & \hspace*{4.5in} \\
    AFP  & \dotfill & Anterior Forebrain Pathway \\
    BMI   & \dotfill & Brain Machine Interface \\
    CAD  & \dotfill & Computer-Aided Design \\
    DLM & \dotfill & dorsolateral anterior thalamic nucleus \\
    FWHM & \dotfill & Full-Width at Half Maximum \\
    CNC & \dotfill & Complementary metal-oxide-semiconductor \\
    HVC & \dotfill & Computer numerical control  \\
    HVC$_{RA}$  & \dotfill & HVC neuron that projects to RA \\
	HVC$_{X}$  & \dotfill & HVC neuron that projects to Area X\\
	HVC$_{AV}$  & \dotfill & HVC neuron that projects to Avalanche \\
	HVC$_{I}$  & \dotfill & HVC interneuron\\
    ISIH  & \dotfill & Inter-spike interval histogram\\
    LFP  & \dotfill & Local Field Potential \\
    LMAN  & \dotfill & lateral magnocellular nucleus of the anterior neostriatum \\
    RSV & \dotfill &  Rous sarcoma virus \\
    SNR  & \dotfill & signal-to-noise ratio \\
    SD  & \dotfill & standard deviation \\
    STL & \dotfill & STereoLithography \\
    UVA   & \dotfill & nucleus uvaeformis \\
    VMP  & \dotfill & vocal-motor pathway \\
    
  \end{tabular}
\end{center}
\cleardoublepage

% END OF THE PRELIMINARY PAGES

\newpage
\endofprelim
